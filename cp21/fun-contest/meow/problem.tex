\input{.template/template.tex}

\begin{document}

\makeheader{end of contest}

During social distancing, you and your cat spent way too much time together. It is very annoyed by your presence by now and it's \textit{meow}s are getting furious.
To make it less difficult to live with your grumpy cat, you wanted to find a way to figure out when it is extremely unfriendly and annoyed, so you can take precautions.
Many furious \textit{meow}s later, you started to notice a pattern.


When you add the number of \textit{e}'s and \textit{o}'s in the \textit{meow} of your cat together, you can calculate its happiness by counting all possible ways this number \textit{n} can be expressed as the sum of consecutive integers larger than zero.

\paragraph*{Input}

The input starts with a line containing a single integer \textit{t} 
\begin{math}(1\leq t\leq 10^4)\end{math}
 the number of test cases that follow.

Each test case consists of one line, each containing 1 string \textit{s}. The length of the string is between\begin{math}(4\leq length\leq 2*10^6+2)\end{math}.
The string is the \textit{meow} of your cat, that is guaranteed to have each of the letters m, e, o, w at least one time.
\paragraph*{Output}

For each test case print the politeness of your cat and it's \textit{meow}.
\begin{samples}
  \sample{sample1}
\end{samples}

\paragraph*{Sample Description}
The second test case in the given sample contains 9 \textit{e}'s and \textit{o}'s. For the number 9, you can find 2 different represeantions as consecutive integers.\\
9 = 2 + 3 + 4\\
9 = 4 + 5

\end{document}